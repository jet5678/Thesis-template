\chapter{Einleitung}
\label{cpt:Einleitung}

\section{Problemstellung}
\label{sec:Problemstellung}

Was soll thematisiert werden?\\
Die Empfohlene Anzahl an Seiten für eine Bachelorarbeit liegt bei etwa 40 und etwa 60 für eine Masterarbeit. Ausführliche Tabellen, Quellcode oder Rohdaten sollten Teil des Anhangs sein um das Lesen des Hauptdokumentes übersichtlicher zu machen. 

\section{Zielsetzung}
\label{sec:Zielsetzung}

In der Regel steht hier so etwas wie die Motivation.

\section{Abgrenzung}
\label{sec:Abgrenzung}

Was wird betrachtet und was nicht, warum wird die Grenze genau an der Stelle gezogen?\\
Dabei sollte die eigene Arbeit auch gegen andere Arbeiten abgegrenzt werden. Hierfür ist eine Literaturrecherche unerlässlich. Für die Literaturrecherche können beispielsweise folgende Suchmaschinen einen Anfangspunkt bilden, diese sind besonders geeignet um Paper und andere wissenschaftliche Veröffentlichungen zu finden, wie z.B. \cite{Hehnen.2020}:
\begin{itemize}[itemsep=-6pt]
	\item \url{https://scholar.google.de/}
	\item \url{https://www.sciencedirect.com/}
	\item \url{https://www.ulb.uni-muenster.de/lotse/}
	\item \url{https://www.springer.com/gp}
\end{itemize}

Weitere Hinweise zu Inhalten von Kapiteln, möglichem Aufbau der Arbeit sowie Hinweise zum wissenschaftlichen Schreiben können der Präsentation zur Vorlesung \glqq Einführung in das wissenschaftliche Schreiben\grqq entnommen werden. Diese steht öffentlich zum Download unter folgender Adresse zur Verfügung.
\url{https://www.asim.uni-wuppertal.de/de/lehre/wissenschaftliches-schreiben.html}

\section{Herangehensweise}
\label{sec:Herangehensweise}

Welche Hilfsmittel werden verwendet z.B. \ac{FDS}, FDSReader, PROPTI.