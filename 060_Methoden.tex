\chapter{Methoden}
\label{cpt:Methoden}

Vorstellung des Standes der Technik und Wissenschaft anhand bereits existierender Normen und wissenschaftlicher Arbeiten. Sowie eine knappe Beschreibung der Methoden, die für diese Arbeit relevant sind und über den durchschnittlichen Kenntnisstand des Lesers hinaus gehen. \\

Bei der Verwendung dieser vierteiligen Gliederung bestehend aus:
\begin{enumerate}[itemsep=-6pt]
	\item Einleitung
	\item Methoden
	\item Ergebnisse
	\item Diskussion
\end{enumerate}
gehören in den Methoden Teil auch die eigenen Arbeiten mit den folgenden Inhalten.
\begin{itemize}[itemsep=-6pt]
	\item Dokumentation der eigenen Arbeiten
	\item Auflistung der verwendeten Methoden und deren Parameter
	\item Verifikation, Validierung und Plausibilität der Methoden und Parameter
	\item Struktur und Vorgehensweise
	\item Exemplarische oder charakteristische Rohdaten präsentieren
\end{itemize}

Dabei werden z.B. die verwendeten Formeln wie hier \autoref{equ:Nav} dargestellt.

\begin{equation}
	\label{equ:Nav}
	\mathrm{\partial_t} \rho \vec{\mathsf{v}} + \nabla \cdot (\rho \vec{\mathsf{v}} \vec{\mathsf{v}}) = -\nabla \mathsf{p} + \mu \nabla^2 \vec{\mathsf{v}} + \vec{\mathsf{f}} 
\end{equation}

Es können auch Aufzählungen verschiedener Arten verwendet werden. Hierbei sollte darauf geachtet werden, dass die Abstände zwischen Punkten der gleichen Gliederungsstufe den gleichen Zeilenabstand haben wie der übrige Text. 

\begin{itemize}[itemsep=-6pt]
	\item erste Ebene
	\begin{itemize}[itemsep=-6pt]
		\item zweite Ebene
		\item auch zweite Ebene
		\begin{itemize}[itemsep=-6pt]
			\item dritte Ebene
			\begin{itemize}[itemsep=-6pt]
				\item vierte Ebene
			\end{itemize}
		\end{itemize}
	\end{itemize}
\end{itemize}

\begin{itemize}[itemsep=-6pt]
	\item[-] ein selbst gewähltes Aufzählungszeichen
	\item[\#] ein anderes selbst gewähltes Aufzählungszeichen
\end{itemize}

